% author Dinupa Nawarathne
% email dinupa3@gmail.com
% date 11-18-2022

\documentclass[10pt, xcolor={dvipsnames}, aspectratio = 169, sans,mathserif]{beamer}

%
% packages
%
\usepackage{fontspec}
\usepackage{fontawesome5}
\usepackage{mathrsfs}
\usepackage{amsmath}
\usepackage{graphicx}
\usepackage{hyperref}
\usepackage[absolute,overlay]{textpos}
\usepackage[font=tiny]{caption}
\usepackage{xcolor,colortbl}


%
% customization
%
\mode<presentation>
{
\usefonttheme{serif}
\setmainfont{JetBrains Mono}
% custom colors
\definecolor{nmsured}{RGB}{137,18,22}
% custom fonts
\setbeamercolor{title}{bg=White,fg=nmsured}
\setbeamercolor{frametitle}{bg=White,fg=nmsured}
\setbeamercolor{section number projected}{bg=nmsured,fg=White}
\setbeamercolor{subsection number projected}{bg=nmsured,fg=White}
\setbeamertemplate{items}{\color{nmsured}{\faAngleDoubleRight}}
\setbeamertemplate{section in toc}[square]
\setbeamertemplate{subsection in toc}[square]
\setbeamertemplate{footline}[frame number]
\setbeamertemplate{caption}[numbered]
\setbeamerfont{footnote}{size=\tiny}
\setbeamercovered{invisible}
\usefonttheme{professionalfonts}
% set background
\setbeamertemplate{background}[grid][color=nmsured!15]

\graphicspath{ {./imgs/} }

}

%
% title, author, date
%
\title{Hodoscope \& NIM Handbook}

\author{NMSU Nuclear Physics Group}

\date{E1039/SpinQuest Experiment
\\ \today}

\titlegraphic{
\includegraphics[width=3.0cm]{Fermilab_logo.svg.png}
\includegraphics[width=1.0cm]{nmsu.png}
\includegraphics[width=1.5cm]{spinquest_logo.png}
\includegraphics[width=1.3cm]{Seal_of_the_United_States_Department_of_Energy.svg.png}
}

%
% some custom commands
%
\newcommand{\leftpic}[2]
{
\begin{textblock}{7.0}(0.5, 1.5)
\begin{figure}
    \centering
    \includegraphics[width=7.0cm]{#1}
    \caption{#2}
\end{figure}
\end{textblock}
}

\newcommand{\rightpic}[2]
{
\begin{textblock}{7.0}(8.0, 1.5)
\begin{figure}
    \centering
    \includegraphics[width=7.0cm]{#1}
    \caption{#2}
\end{figure}
\end{textblock}
}


\begin{document}

% make title page
\begin{frame}
    \maketitle
\end{frame}


% slide 1
\begin{frame}[fragile]{SpinQuest Spectrometer Layout}

\begin{footnotesize}
\begin{table}
\begin{center}
\begin{tabular}{ c c c c c } 
  & & Beam dump & & \\
  & & ST1 wire chamber & & \\
  & & \cellcolor{red!25} H1Y & & \cellcolor{red!25} H1 HV \\
  & & \cellcolor{red!25} H1X & & \\
  & & & & \\
  & & DP1 & \\
  & & KMag & \\
\cellcolor{red!25} H2 HV & & & &  \\
\cellcolor{red!25} NIM H1  (RACK 27) & & ST2 Wire chambers  & & \\
\cellcolor{red!25} NIM H2 (RR10) & & \cellcolor{red!25} H2Y  & & \\
 & & \cellcolor{red!25} H2X & & \\
  & & \cellcolor{red!25} DP2 & & \\
\cellcolor{red!25} FPGA (RACK 16) & & & & \\
\cellcolor{red!25} RF coincidence (RR 42) & & ST3 Wire chambers & & \\
 & & \cellcolor{red!25} H3X & & \\
\cellcolor{red!25} H3/H4 HV & & Hadron absorber & & \\
\cellcolor{red!25} NIM H3/H4(RR83) & & PT1 & & \\
\cellcolor{red!25} NIM H3/H4 (RR43) & & \cellcolor{red!25} H4Y1 && \\
 & & PT2 & & \\
 & & \cellcolor{red!25} H4Y2 & & \\
  & & \cellcolor{red!25} H4X & & 
\end{tabular}
\caption{Block diagram of the SpinQuest spectrometer. Pink shaded blocks are in the Hodoscope/NIM sub-system.}
\end{center}
\end{table}
\end{footnotesize}

\end{frame}

\begin{frame}{Debugging Hodoscope/NIM Sub-System}

Some common steps of debugging Hodoscope/NIM sub-system includes;

\begin{itemize}

    \item Check the hodocope hits in online monitor.

    \item Check the raw signal from the hodoscopes.

    \item Check the output signal from the discriminator.

    \item Check the raw NIM signal.

    \item Check the RF gated signal.

\end{itemize}

Order of these steps might change depending on the type of the issue.

\end{frame}

\begin{frame}{Replacing Electronics}

\begin{itemize}

    \item When replacing 4413 discriminators make sure to;

    \begin{itemize}

        \item keep the module threshold at the minimum. Typically this is $\sim$ -15 mV.

        \item adjust the width to $\sim$ 15 - 20 ns. 

    \end{itemize}

    \item When replacing LEMO cables (or any cables) make sure that they are in the same length and they have correct labels in the both ends of the cable.

\end{itemize}

\end{frame}


% slide 2
\begin{frame}{NIM Signal}

\leftpic{signal_1.jpg}{Typical raw signal from the hodoscope paddle.}

\rightpic{signal_2.jpg}{Typical NIM output signal from the LeCroy 4413 discriminator. The threshold of the discriminator set for the minimum value ($\sim$-15mV) and the width is set to $\sim$ 15ns.}

\end{frame}

\begin{frame}{H1 Patch Panels}

\begin{table}
\begin{center}
\begin{tabular}{ | c | c | c | c | c | c | c | c | c | c |}
\hline
H1T23 & H1T11 & H1B23 & H1B11 & & & H1L20 & H1R20 & H1L10 & H1R 10 \\
\hline
H1T22 & H1T10 & H1B22 & H1B10 & & & H1L19 & H1R19 & H1L09 & H1R 09 \\
\hline
H1T21 & H1T09 & H1B21 & H1B09 & & & H1L18 & H1R18 & H1L08 & H1R 08 \\
\hline
H1T20 & H1T08 & H1B20 & H1B08 & & & H1L17 & H1R17 & H1L07 & H1R 07 \\
\hline
H1T19 & H1T07 & H1B19 & H1B07 & & & H1L16 & H1R16 & H1L06 & H1R 06 \\
\hline
H1T18 & H1T06 & H1B18 & H1B06 & & & & & &  \\
\hline
\end{tabular}
\caption{Block diagram of the top patch panel.}
\end{center}
\end{table}

\begin{table}
\begin{center}
\begin{tabular}{ | c | c | c | c | c | c | c | c | c | c |}
\hline
H1T17 & H1T05 & H1B17 & H1B05 & & & H1L15 & H1R15 & H1L05 & H1R 05 \\
\hline
H1T16 & H1T04 & H1B16 & H1B04 & & & H1L14 & H1R14 & H1L04 & H1R 04 \\
\hline
H1T15 & H1T03 & H1B15 & H1B03 & & & H1L13 & H1R13 & H1L03 & H1R 03 \\
\hline
H1T14 & H1T02 & H1B14 & H1B02 & & & H1L12 & H1R12 & H1L02 & H1R 02 \\
\hline
H1T13 & H1T01 & H1B13 & H1B01 & & & H1L11 & H1R11 & H1L01 & H1R 01 \\
\hline
H1T12 & & H1B12 & & & & & & &  \\
\hline
\end{tabular}
\caption{Block diagram of the bottom patch panel.}
\end{center}
\end{table}

\end{frame}

\begin{frame}{4413 Pin Numbers}

\begin{table}
\begin{center}
\begin{tabular}{ | c | c |}
\hline
7 & 8 \\
\hline
5 & 6 \\
\hline
3 & 4 \\
\hline
1 & 2 \\
\hline
 & \\
\hline
9 & 10 \\
\hline
11 & 12 \\
\hline
13 & 14 \\
\hline
15 & 16 \\
\hline
\end{tabular}
\caption{Pin numbers of the 4413 discriminator.}
\end{center}
\end{table}

\end{frame}

\begin{frame}{H1 4413 Discriminator Channel Mapping}

\begin{textblock}{14.0}(0.5, 2.0)
\begin{table}
%\begin{center}
\begin{tabular}{ | c | c | c | c | c | c | c | c | c | c | c | c | c |}
\hline
H1XT13 & H1XT12 & & H1XB13 & H1XB12 & & H1XT02 & H1XT01 & & H1YL17 & H1YL16 \\
\hline
H1XT15 & H1XT14 & & H1XB15 & H1XB14 & & H1XT04 & H1XT03 & & H1YL19 & H1YL18 \\
\hline
H1XT17 & H1XT16 & & H1XB17 & H1XB16 & & H1XT21 & H11XT20 & & H1YL20 & \\
\hline
H1XT19 & H1XT18 & & H1XB19 & H1XB18 & & H1XT23 & H1XT22 & & & \\
\hline
 & & & & & & & & & & \\
\hline
H1XT11 & H1XT10 & & H1XB11 & H1XB10 & & H1XB23 & H1XB22 & & H1YL15 & H1YL14 \\
\hline
H1XT09 & H1XT08 & & H1XB09 & H1XB08 & & H1XB21 & H1XB20 & & H1YL13 & H1YL12 \\
\hline
H1XT07 & H1XT06 & & H1XB07 & H1XB06 & & H1XB04 & H1XB03 & & H1YL11 & \\
\hline
H1XT05 &  &  & H1XB05 & & & H1XB02 & H1XB01 & & & \\
\hline
\end{tabular}
%\caption{Pin numbers of the 4413 discriminator.}
%\end{center}
\end{table}
\end{textblock}

\end{frame}

\begin{frame}{H1 4413 Discriminator Channel Mapping}
 

\end{frame}

\end{document}