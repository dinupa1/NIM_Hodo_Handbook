% author Dinupa Nawarathne
% email dinupa3@gmail.com
% date 11-18-2022

\documentclass[10pt, xcolor={dvipsnames}, aspectratio = 169, sans,mathserif]{beamer}

%
% packages
%
\usepackage{fontspec}
\usepackage{fontawesome5}
\usepackage{mathrsfs}
\usepackage{amsmath}
\usepackage{graphicx}
\usepackage{hyperref}
\usepackage[absolute,overlay]{textpos}
\usepackage[font=tiny]{caption}
\usepackage{xcolor,colortbl}


%
% customization
%
\mode<presentation>
{
\usefonttheme{serif}
\setmainfont{JetBrains Mono}
% custom colors
\definecolor{nmsured}{RGB}{137,18,22}
% custom fonts
\setbeamercolor{title}{bg=White,fg=nmsured}
\setbeamercolor{frametitle}{bg=White,fg=nmsured}
\setbeamercolor{section number projected}{bg=nmsured,fg=White}
\setbeamercolor{subsection number projected}{bg=nmsured,fg=White}
\setbeamertemplate{items}{\color{nmsured}{\faAngleDoubleRight}}
\setbeamertemplate{section in toc}[square]
\setbeamertemplate{subsection in toc}[square]
\setbeamertemplate{footline}[frame number]
\setbeamertemplate{caption}[numbered]
\setbeamerfont{footnote}{size=\tiny}
\setbeamercovered{invisible}
\usefonttheme{professionalfonts}
% set background
\setbeamertemplate{background}[grid][color=nmsured!15]

\graphicspath{ {./imgs/} }

}

%
% title, author, date
%
\title{Hodoscope \& NIM Handbook}

\author{NMSU Nuclear Physics Group}

\date{E1039/SpinQuest Experiment
\\ \today}

\titlegraphic{
\includegraphics[width=3.0cm]{Fermilab_logo.svg.png}
\includegraphics[width=1.0cm]{nmsu.png}
\includegraphics[width=1.5cm]{spinquest_logo.png}
\includegraphics[width=1.3cm]{Seal_of_the_United_States_Department_of_Energy.svg.png}
}

%
% some custom commands
%
\newcommand{\leftpic}[2]
{
\begin{textblock}{7.0}(0.5, 1.5)
\begin{figure}
    \centering
    \includegraphics[width=7.0cm]{#1}
    \caption{#2}
\end{figure}
\end{textblock}
}

\newcommand{\rightpic}[2]
{
\begin{textblock}{7.0}(8.0, 1.5)
\begin{figure}
    \centering
    \includegraphics[width=7.0cm]{#1}
    \caption{#2}
\end{figure}
\end{textblock}
}


\begin{document}

% make title page
\begin{frame}
    \maketitle
\end{frame}


% slide 1
\begin{frame}[fragile]{SpinQuest Spectrometer Layout}

\begin{footnotesize}
\begin{table}
\begin{center}
\begin{tabular}{ c c c c c } 
  & & Beam dump & & \\
  & & ST1 wire chamber & & \\
  & & \cellcolor{red!25} H1Y & & \cellcolor{red!25} H1 HV \\
  & & \cellcolor{red!25} H1X & & \\
  & & & & \\
  & & DP1 & \\
  & & KMag & \\
\cellcolor{red!25} H2 HV & & & &  \\
\cellcolor{red!25} NIM H1  (RACK 27) & & ST2 Wire chambers  & & \\
\cellcolor{red!25} NIM H2 (RR10) & & \cellcolor{red!25} H2Y  & & \\
 & & \cellcolor{red!25} H2X & & \\
  & & \cellcolor{red!25} DP2 & & \\
\cellcolor{red!25} FPGA (RACK 16) & & & & \\
\cellcolor{red!25} RF coincidence (RR 42) & & ST3 Wire chambers & & \\
 & & \cellcolor{red!25} H3X & & \\
\cellcolor{red!25} H3/H4 HV & & Hadron absorber & & \\
\cellcolor{red!25} NIM H3/H4(RR83) & & PT1 & & \\
\cellcolor{red!25} NIM H3/H4 (RR43) & & \cellcolor{red!25} H4Y1 && \\
 & & PT2 & & \\
 & & \cellcolor{red!25} H4Y2 & & \\
  & & \cellcolor{red!25} H4X & & 
\end{tabular}
\caption{Block diagram of the SpinQuest spectrometer. Pink shaded blocks are in the Hodoscope/NIM sub-system.}
\end{center}
\end{table}
\end{footnotesize}

\end{frame}

\begin{frame}{Debugging Hodoscope/NIM Sub-System}

Some common steps of debugging Hodoscope/NIM sub-system includes;

\begin{itemize}

    \item Check the hodocope hits in online monitor.

    \item Check the raw signal from the hodoscopes.

    \item Check the output signal from the discriminator.

    \item Check the raw NIM signal.

    \item Check the RF gated signal.

\end{itemize}

Order of these steps might change depending on the type of the issue.

\end{frame}

\begin{frame}{Replacing Electronics}

\begin{itemize}

    \item When replacing 4413 discriminators make sure to;

    \begin{itemize}

        \item keep the module threshold at the minimum. Typically this is $\sim$ -15 mV.

        \item adjust the width to $\sim$ 15 - 20 ns. 

    \end{itemize}

    \item When replacing LEMO cables (or any cables) make sure that they are in the same length and they have correct labels in the both ends of the cable.

\end{itemize}

\end{frame}


% slide 2
\begin{frame}{NIM Signal}

\leftpic{signal_1.jpg}{Typical raw signal from the hodoscope paddle. Time scale 20ns/div. and voltage scale 500mV/div.}

\rightpic{signal_2.jpg}{Typical NIM output signal from the LeCroy 4413 discriminator. The threshold of the discriminator set for the minimum value ($\sim$-15mV) and the width is set to $\sim$ 15ns. Time scale 10ns/div. and voltage scale 500mV/div.}

\end{frame}

\begin{frame}[fragile]{Electronics}

\begin{textblock}{14.0}(1.0, 2.0)

Mostly used electronics in the NIM sub-system include.

\begin{itemize}

    \item LeCroy Model 4413: a 16-input discriminator in a single- width CAMAC module. \href{https://hallcweb.jlab.org/experiments/hks/datasheets/4413-spec.htm}{Info.}

    \item LeCroy Model 623B: a low cost eight channel discriminator featuring high-sensitivity, high-speed, updating performance. \href{https://www.fnal.gov/projects/ckm/jlab/623b-spec.htm}{Info.}

    \item LeCroy Model 4416 ECL/NIM/ECL: The Model 4416 is simultaneously an ECL-to-NIM and a NIM-to-ECL converter, specially designed to fill the gap between the new growing ECL circuitry and the old NIM electronics. \href{https://www.fnal.gov/projects/ckm/jlab/4616-spec.htm}{Info.}

    \item LeCroy Model 1440 system: This is the high voltage system for the hodoscopes. \href{https://teledynelecroy.com/lrs/dsheets/dslib.htm}{Info.}

    \item For general infomations about NIM electronics \href{https://teledynelecroy.com/lrs/dsheets/dslib.htm}{link}.

\end{itemize}

\end{textblock}

\end{frame}

\begin{frame}{H1 Patch Panels}

\begin{table}
\begin{center}
\begin{tabular}{ | c | c | c | c | c | c | c | c | c | c |}
\hline
H1T23 & H1T11 & H1B23 & H1B11 & & & H1L20 & H1R20 & H1L10 & H1R 10 \\
\hline
H1T22 & H1T10 & H1B22 & H1B10 & & & H1L19 & H1R19 & H1L09 & H1R 09 \\
\hline
H1T21 & H1T09 & H1B21 & H1B09 & & & H1L18 & H1R18 & H1L08 & H1R 08 \\
\hline
H1T20 & H1T08 & H1B20 & H1B08 & & & H1L17 & H1R17 & H1L07 & H1R 07 \\
\hline
H1T19 & H1T07 & H1B19 & H1B07 & & & H1L16 & H1R16 & H1L06 & H1R 06 \\
\hline
H1T18 & H1T06 & H1B18 & H1B06 & & & & & &  \\
\hline
\end{tabular}
\caption{Block diagram of the top patch panel.}
\end{center}
\end{table}

\begin{table}
\begin{center}
\begin{tabular}{ | c | c | c | c | c | c | c | c | c | c |}
\hline
H1T17 & H1T05 & H1B17 & H1B05 & & & H1L15 & H1R15 & H1L05 & H1R 05 \\
\hline
H1T16 & H1T04 & H1B16 & H1B04 & & & H1L14 & H1R14 & H1L04 & H1R 04 \\
\hline
H1T15 & H1T03 & H1B15 & H1B03 & & & H1L13 & H1R13 & H1L03 & H1R 03 \\
\hline
H1T14 & H1T02 & H1B14 & H1B02 & & & H1L12 & H1R12 & H1L02 & H1R 02 \\
\hline
H1T13 & H1T01 & H1B13 & H1B01 & & & H1L11 & H1R11 & H1L01 & H1R 01 \\
\hline
H1T12 & & H1B12 & & & & & & &  \\
\hline
\end{tabular}
\caption{Block diagram of the bottom patch panel.}
\end{center}
\end{table}

\end{frame}

\begin{frame}{4413 Pin Numbers}

\begin{textblock}{7.0}(0.5, 2.0)
\begin{table}
\begin{center}
\begin{tabular}{ | c | c |}
\hline
7 & 8 \\
\hline
5 & 6 \\
\hline
3 & 4 \\
\hline
1 & 2 \\
\hline
 & \\
\hline
9 & 10 \\
\hline
11 & 12 \\
\hline
13 & 14 \\
\hline
15 & 16 \\
\hline
\end{tabular}
\caption{Pin numbers of the 4413 discriminator.}
\end{center}
\end{table}
\end{textblock}

\begin{textblock}{7.0}(8.0, 1.5)
\begin{figure}
    \centering
    \includegraphics[height=7.0cm]{4413.jpg}
    \caption{A 4413 discriminator pins.}
\end{figure}
\end{textblock}

\end{frame}

\begin{frame}{H1 4413 Discriminator Channel Mapping}

\begin{textblock}{14.0}(0.5, 2.0)
\begin{table}
\begin{center}
\begin{tabular}{ 
                |>{\columncolor{blue!20}} c
                >{\columncolor{blue!20}} c
                |>{\columncolor{red!20}} c
                >{\columncolor{red!20}} c
                |>{\columncolor{green!20}} c
                >{\columncolor{green!20}}c |
                }
\hline
H1XT13 & H1XT12 & H1XB13 & H1XB12 & H1XT02 & H1XT01 \\
H1XT15 & H1XT14 & H1XB15 & H1XB14 & H1XT04 & H1XT03 \\
H1XT17 & H1XT16 & H1XB17 & H1XB16 & H1XT21 & H11XT20 \\
H1XT19 & H1XT18 & H1XB19 & H1XB18 & H1XT23 & H1XT22  \\
 & & & & &  \\
H1XT11 & H1XT10 & H1XB11 & H1XB10 & H1XB23 & H1XB22 \\
H1XT09 & H1XT08 & H1XB09 & H1XB08 & H1XB21 & H1XB20 \\
H1XT07 & H1XT06 & H1XB07 & H1XB06 & H1XB04 & H1XB03 \\
H1XT05 &  &   H1XB05 & &  H1XB02 & H1XB01 \\
\hline
\end{tabular}
\caption{H1X channel mapping to discriminators. This is the layout of the left most 3 4413 16 channel discriminators in the NIM bin. Each color represent one discriminator.}
\end{center}
\end{table}
\end{textblock}

\end{frame}

\begin{frame}{H1 4413 Discriminator Channel Mapping}

\begin{table}
\begin{center}
\begin{tabular}{
    |>{\columncolor{blue!20}} c
    >{\columncolor{blue!20}} c
    |>{\columncolor{red!20}} c
    >{\columncolor{red!20}} c
    |>{\columncolor{green!20}} c
    >{\columncolor{green!20}} c
    |>{\columncolor{orange!20}} c
    >{\columncolor{orange!20}} c|
}
\hline
H1YL17 & H1YL16 & H1YR17 & H1YR16 & H1YL07 & H1YL06 & H1YR07 & H1YR06 \\
H1YL19 & H1YL18 & H1YR19 & H1YR18 & H1YL09 & H1YL08 & H1YR09 & H1YR08 \\
H1YL20 &        &        & H1YR20 &        & H1YL10 &        & H1YR10 \\
       &        &        &        &        &        &        &        \\
       &        &        &        &        &        &        &        \\
H1YL15 & H1YL14 & H1YR15 & H1YR14 & H1YL05 & H1YL04 & H1YR05 & H1YR04 \\
H1YL13 & H1YL12 & H1YR13 & H1YR12 & H1YL03 & H1YL02 & H1YR03 & H1YR02 \\
H1YL11 &        & H1YR11 &        & H1YL01 &        & H1YR02 &        \\
       &        &        &        &        &        &        &        \\
\hline
\end{tabular}
\caption{H1Y channel mapping to discriminators. This is the layout of the right most 4 4413 16 channel discriminators in the NIM bin. Each color represent one discriminator.}
\end{center}
\end{table}

\end{frame}

\begin{frame}{H2 Patch Panel Layout}
\begin{table}
\begin{center}
\begin{tabular}{|c|c|c|c|c|}
\hline
H2L19 & H2R19 & H2L09 & H2T16 & H2B16 \\
\hline
H2L18 & H2R18 & H2L08 & H2T15 & H2B15 \\
\hline
H2L17 & H2R17 & H2L07 & H2T14 & H2B14 \\
\hline
H2L16 & H2R16 & H2L06 & H2T13 & H2B13 \\
\hline
H2L15 & H2R15 & H2L05 & H2T12 & H2B12 \\
\hline
      & H2R09 & H2R06 & H2T11 & H2B11 \\
\hline
      & H2R08 & H2R05 & H2T10 & H2B10 \\
\hline
      & H2R07 &       & H2T09 & H2B09 \\
\hline
\end{tabular}
\caption{H2 top patch panel. This is located in rack RR10.}
\end{center}
\end{table}

\begin{tiny}
\begin{table}
\begin{center}
\begin{tabular}{|c|c|c|c|c|c|c|c|c|c|c|c|c|c|c|}
\hline
H2L14 & H2L11 & H2R13 & H2L04 & H2L02 & H2R04 & H2R02 & H2T08 & H2T06 & H2T04 & H2T02 & H2B08 & H2B06 & H2B04 & H2B02 \\
\hline
H2L13 & H2L10 & H2R12 & H2L03 & H2L01 & H2R03 & H2R01 & H2T07 & H2T05 & H2T03 & H2T01 & H2B07 & H2B05 & H2B03 & H2B01 \\
\hline
H2L12 & H2R14 & H2R11 & H2R10 &       &       &       &       &       &       &       &       &       &       &       \\
\hline
\end{tabular}
\caption{H2 bottom patch panel. This is located in rack RR10.}
\end{center}
\end{table}
\end{tiny}

\end{frame}

\begin{frame}{H2 Amplifier}

\begin{table}
\begin{center}
\begin{tabular}{|c|c|c|c|c|c|c|c|c|}
\hline
H2L19 & H2L18 & H2L17 & H2L09 & H2L08 & H2B16 & H2B15 & H2B15 & H2B14 \\
\hline
H2L16 & H2L15 & H2R19 & H2L06 & H2L05 & H2B13 & H2B12 & H2B12 & H2B11 \\
\hline
H2R18 & H2R17 & H2R16 & H2R08 & H2R07 & H2B10 & H2B09 & H2B09 & H2T12 \\
\hline
H2R15 & H2T16 & H2T15 & H2R05 & H2T14 & H2T13 & H2T11 & H2T10 & H2T09 \\
\hline
\end{tabular}
\caption{H2 top amplifiers. This is located in rack RR10.}
\end{center}
\end{table}

\begin{table}
\begin{center}
\begin{tabular}{|c|c|c|c|c|c|c|c|c|}
\hline
H2L14 & H2L13 & H2L12 & H2L04 & H2L03 & BAD   & H2T04 & H2T03 & H2T02 \\
\hline
H2L11 & H2L10 & H2L02 & H2L01 & H2R04 & H2R03 & H2T01 & H2B08 & H2B07 \\
\hline
BAD   & H2R13 & H2R12 & H2R02 & H2R01 & H2T08 & H2B06 & H2B05 & H2B04 \\
\hline
H2R11 & H2R10 & H2R14 & H2T07 & H2T07 & H2T05 & H2B03 & H2B02 & H2B01 \\
\hline
\end{tabular}
\caption{H2 bottom amplifiers. This is located in rack RR10.}
\end{center}
\end{table}

\end{frame}

\begin{frame}{H2 4413 Discriminator Mapping}

\begin{textblock}{14.0}(0.5, 2.0)
\begin{table}
\begin{center}
\begin{tabular}{
    |>{\columncolor{lime!20}}c
    >{\columncolor{lime!20}}c
    |>{\columncolor{blue!20}}c
    >{\columncolor{blue!20}}c
    |>{\columncolor{green!20}}c
    >{\columncolor{green!20}}c
    |>{\columncolor{red!20}}c
    >{\columncolor{red!20}}c
    |>{\columncolor{orange!20}}c
    >{\columncolor{orange!20}}c|
}
\hline
H2L16 & H2L15 & H2R16 & H2R15 & H2L06 & H2L05 & H2T19 & H2T09 & H2B10 & H2B09 \\
H2L18 & H2L17 & H2R18 & H2R17 & H2L08 & H2L07 & H2T12 & H2T11 & H2B12 & H2B11 \\
      & H2L19 &       & H2R19 &       & H2L09 & H2T14 & H2T13 & H2B14 & H2B13 \\
      &       &       &       &       &       & H2T16 & H2T15 & H2B16 & H2B15 \\
      &       &       &       &       &       &       &       &       &       \\
      &       &       &       &       &       &       &       &       &       \\
H2L14 & H2L13 & H2R14 & H2R13 & H2L04 & H2R03 & H2T08 & H2T07 & H2B08 & H2B07 \\
H2L12 & H2L11 & H2R12 & H2R11 & H2L02 & H2R01 & H2T06 & H2T05 & H2B06 & H2B05 \\
H2L10 &       & H2R10 &       &       &       & H2T04 & H2T03 & H2B04 & H2B03 \\
      &       &       &       &       &       & H2T02 & H2T01 & H2B02 & H2B01 \\
\hline
\end{tabular}
\caption{H1Y channel mapping to discriminators. Each color represent one discriminator. This is located at RR10 rack.}
\end{center}
\end{table}
\end{textblock}
\end{frame}

\begin{frame}{NIM Trigger Mapping}

\begin{textblock}{7.0}(0.5, 2.0)
\begin{table}
\begin{center}
\begin{tabular}{|c|c|c|}
\hline
\rowcolor{red!20}
CHR01 & H1XT & OUT \\
\rowcolor{red!20}
CHR02 & H1XB & OUT \\
\rowcolor{red!20}
CHR03 & H1YB & OUT \\
\rowcolor{red!20}
CHR04 & H1YT & OUT \\
\rowcolor{red!20}
CHR05 & H2XT & OUT \\
\rowcolor{red!20}
CHR06 & H2XB & OUT \\
\rowcolor{red!20}
CHR07 & H2YB & OUT \\
\rowcolor{red!20}
CHR08 & H2YT & OUT \\
\rowcolor{red!20}
CHR09 & H3XT & OUT \\
\rowcolor{red!20}
CHR10 & H3XB & OUT \\
\rowcolor{red!20}
CHR11 & H4XT & OUT \\
\rowcolor{red!20}
CHR12 & H4XB & OUT \\
\rowcolor{red!20}
CHR13 & H4Y1B & OUT \\
\rowcolor{red!20}
CHR14 & H4Y1T & OUT \\
\rowcolor{red!20}
CHR15 & H4Y2B & OUT \\
\rowcolor{red!20}
CHR16 & H2Y2T & OUT \\
\hline
\end{tabular}
\end{center}
\end{table}
\end{textblock}

\begin{textblock}{7.0}(8.0, 2.0)
\begin{table}
\begin{center}
\begin{tabular}{|c|c|c|}
\hline
CHR17 &       &     \\
CHR18 &       &     \\
CHR19 &       &     \\
CHR20 &       &     \\
\rowcolor{lime!20}
CHR21 & NIM1 & IN \\
\rowcolor{lime!20}
CHR22 & NIM2 & IN \\
\rowcolor{lime!20}
CHR23 & NIM4 & IN \\
\rowcolor{lime!20}
CHR24 & MATRIX & IN \\
\hline
\end{tabular}
\caption{NIM trigger channel mapping. This is located at RR42 rack.}
\end{center}
\end{table}
\end{textblock}

\end{frame}

\begin{frame}{H4 4413 Discriminator Channel Mapping}

\begin{textblock}{7.0}(0.5, 2.0)
\begin{table}
\begin{center}
\begin{tabular}{
    |>{\columncolor{lime!20}}c
    |>{\columncolor{blue!20}}c
    |>{\columncolor{red!20}}c|
}
\hline
CHANNEL & ELEMENT & ELEMENT \\
\hline
8 & 1 & 9 \\
7 & 2 & 10 \\
6 & 3 & 11 \\
5 & 4 & 12 \\
4 & 5 & 13 \\
3 & 6 & 14 \\
2 & 7 & 15 \\
1 & 8 & 16 \\
\hline
\end{tabular}
\caption{H4 up PMT channel mapping to 4413 discriminators. Each color represent one discriminator. This is located at RR83 rack.}
\end{center}
\end{table}
\end{textblock}

\begin{textblock}{7.0}(8.0, 2.0)
\begin{table}
\begin{center}
\begin{tabular}{
    |>{\columncolor{lime!20}}c
    |>{\columncolor{blue!20}}c
    |>{\columncolor{red!20}}c|
}
\hline
CHANNEL & ELEMENT & ELEMENT \\
\hline
16 & 1 & 9 \\
15 & 2 & 10 \\
14 & 3 & 11 \\
13 & 4 & 12 \\
12 & 5 & 13 \\
11 & 6 & 14 \\
10 & 7 & 15 \\
9  & 8 & 16 \\
\hline
\end{tabular}
\caption{H4 up PMT channel mapping to 4413 discriminators. Each color represent one discriminator. This is located at RR83 rack.}
\end{center}
\end{table}
\end{textblock}

\end{frame}

\begin{frame}[fragile]{NIM Triggers}

\begin{itemize}
    \item Four NIM triggers were used with cosmic rays;
\end{itemize}

\begin{verbatim}
NIM1: (H1X||H1Y)&&(H2X||H2Y)&&(H3X||H4Y1)&&(H4X||H4Y2)
NIM2: (H1X||H1Y)&&(H2X||H2Y)
NIM3: Random
NIM4: (H2X||H2Y)&&(H4X||H4Y2)
FPGA5: ((H1X||H1Y)&&(H2X||H2Y)) || ((H2X||H2Y)&&(H4X||H4Y2)) // Made
with NIM modules. Time adjustment is reversed beamlike. RF timing is
not included.
\end{verbatim}

\begin{itemize}
    \item Four fold NIM trigger will use in the beam time;
\end{itemize}

\begin{verbatim}
TRIGGER: H1X && H2X && H3X && H4X
\end{verbatim}

\end{frame}

\begin{frame}{RF Timed Trigger}

\begin{textblock}{14.0}(0.5, 1.5)
\begin{figure}
    \centering
    \includegraphics[width=13.0cm]{RF_Timed_NIM_trigger.png}
    \caption{RF timed NIM trigger (Thanks to S. Pate).}
\end{figure}
\end{textblock}

\end{frame}

\begin{frame}[fragile]{Hodoscope Voltage Debugging}

\begin{itemize}

    \item Navigate to settings folder use the following commands;

    \begin{verbatim}
    kinit user@FNAL.GOV
    ssh -X user@spinquestgpvm01.fnal.gov
    cd /data2/e1039/daq/slowcontrols/lecroy/hv/
    \end{verbatim}

    \item Use the following commands for basic debugging;

    \begin{verbatim}
    ./status plane # check status of plane
    ./on plane # turn on the hodoscope plane
    ./off plane # turn off the hodoscope plane
    ./monitor plane # check the status of the plane
    ./reset_netrs232.sh -y plane # reset the connection
    \end{verbatim}

\end{itemize}

\end{frame}

\begin{frame}{Hodoscope Efficiency Software}

\begin{itemize}

    \item Efficiency of the each hodoscope paddle can be calculated using finding the straight line fit. Git \href{https://github.com/forhadnmsu/e1039-analysis/tree/hodo-hvScan/AnaRealDst}{repository}.

    \item Using reconstructed elements use \href{https://github.com/forhadnmsu/e1039-analysis/tree/ktracker-eff/AnaRealDst}{repository}.

    \item Follow the instructions in README file to calculate the efficiencies.

\end{itemize}

\end{frame}

\begin{frame}{Updating this Notebook}

\begin{itemize}

    \item You can update the notebook my making a pull request to the git \href{https://github.com/dinupa1/NIM_Hodo_Handbook}{repository}.

\end{itemize}

\end{frame}

\end{document}